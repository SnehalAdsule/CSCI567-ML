\documentclass[10pt,letterpaper]{article}
\usepackage{amsmath}
\begin{document}
\title{CSCI 567 Assignment 6 \\Fall 2016}
\date{ November 22,2016}
\author{Snehal Adsule\\2080872073\\adsule@usc.edu}
\maketitle
%---- PROBLEM 1
\section{ 1 PCA}
\subsection{1.1 (a) }
\begin{align*}
	J &= \frac{1}{N} \sum_{i=1}^N (x_i -p_{i1}e_i -p_{i2}e_2)^T(x_i -p_{i1}e_i -p_{i2}e_2)\\
	&= \frac{1}{N} \sum_{i=1}^N (x_i^2 - 2p_{i1}e_1x_i - 2p_{i2}e_2x_i -2p_{i1}p_{i2}e_1^Te_2 + p_{i1}^2e_1^Te_1 + p_{i2}^2e_2^Te_2)
	\end{align*}
	Given that $||e1||_2 =1, ||e2||_2 = 1, and e_1^Te_2 = 0,$
	\begin{align*}
J&= \frac{1}{N} \sum_{i=1}^N (x_i^2 - 2p_{i1}e_1^Tx_i - 2p_{i2}e_2^Tx_i - 0 + p_{i1}^2.1 + p_{i2}^2.1)	\\
	\frac{\partial J}{\partial p_{i2}} &=  \frac{1}{N} \sum_{i=1}^N (0- 0-2e_2^Tx_i +0 +2p_{i2})\\
	&= \frac{1}{N} \sum_{i=1}^N (-2e_2^Tx_i+2p_{i2})
	\end{align*}
Setting to zero
	\begin{align*}
		 \frac{1}{N} \sum_{i=1}^N (-2e_2^Tx_i+2p_{i2})=0\\
		\implies   p_{i2} = e_2^Tx_i  \text{ $\forall$ i }
	\end{align*}
\subsection{1.1 (b)}
\begin{align*}
		\tilde{J} &= -e_2^TSe_2 + \lambda_2(e_2^Te_2-1)+\lambda_{12}(e_2^Te_1-0)\\
		\frac{\partial \tilde{J}}{\partial e_2} &= -(S+S^T)e_2 + 2 \lambda_2 e_2 + \lambda_{12}e_1\ \text{given property}\\
		&=-2Se_2+2\lambda_2e_2+\lambda_{12}e_1\ \text{given $S=S^T$}
\end{align*}
		Setting derivative to zero
\begin{align*}
\frac{\partial \tilde{J}}{\partial e_2} = 0\\
			&\implies -2Se_2+2\lambda_2e_2+\lambda_{12}e_1 = 0  \ \text{ we multiply with $e_1^T$}\\
			&\implies -2e_1^TSe_2+2\lambda_2 e_1^Te_2 + \lambda_{12} e_1^Te_1 = 0\\
			&\implies -2(Se_1)^Te_2+ 2 \lambda_2\times 0 + \lambda_{12} \times 1 = 0\ \text{since $S=S^T$}\\
			&  \ \text{As $ (Se_1)^Te_2 = 0$} \implies \lambda_{12} = 0\\
			&\implies Se_2 = \lambda_2e_2   \ \text{by substituting value of $\lambda_{12}$} 
\end{align*}
		Since,  $Se_2 = \lambda_2 e_2$, $e_2$  is the normalized eigenvector associated with the  second largest eigenvector which minimizes $\tilde{J}$ .
\subsection{1.2 (a) Real Example} 
Used the following script to get the eigen values :\\\\
	\textit{
	from numpy import linalg as LA\\
	A=[[91.43 ,171.92, 297.99], [171.92, 373.92 ,545.21], [ 297.99,545.21,1297.26]]\\
	w,v = LA.eig(A)\\
	print w\\
	for i in range(3):\\
    		print 'v',(i+1), '=', v[:,i]
	}\\\\
The eigen values are [ 1626.52644399 ,    7.09745924,   128.98609676]\\
and the three eigen vectors are :\\
v 1 = [ 0.21793758, 0.41449518,  0.88357057] \\
v 2 = [ 0.94428286, -0.31834854, -0.0835709 ] \\
v 3 = [-0.24664366, -0.85255378,  0.46078081]
\subsection{1.2 (b)} 
We can observe that, $\lambda_1$ , $\lambda_3$  and $\lambda_2$ contribute to $92.27\%$  , $7.3175\%$ and $0.40267\%$ of the variance of the bird data. The contribution of $\lambda_2$ is insignificant compared the other two, and can be neglected without much loss of the information after projection of points on the plane formed by the orthogonal vectors $v_1$ and $v_3$.
\subsection{1.2 (c)}  
There are two main directions $v_1$ and $v_3$which captures the birds size with respect to length, wingspan, and weight.  It is intutive that all the weight of the vectors are positive and therefore, shows direct relation between larger length,
wingspan, and weight. Consider first principal component for $v_1$ and we see that feature weight has most impact on the bird's size as corresponsing weight 0.88 is larger than other two features.The $v3$ is the second largest component which is mainly dominated by the wingspan and the length but are in opposite direction with respect to corresponding weight value. 

\section{2.Hidden Markov Model}
\subsection{2 (a)}  
For the gene sequence $O=ACCGTA$, $P(O;\theta) = \sum_{j=1}^2 \alpha_6(j)$\\
Base case: \\
$\alpha_1(j) = P(O_1|S_1=j)P(S_1=j)$ and \\
$ \alpha_t(j) = P(O_t|S_t=j)\sum_{i=1}^2a_{ij}\alpha_{t-1}(j) $  otherwise $(\forall i> 1)$\\
\begin{align*}
&\alpha_1(1) =P(S_1=1)P(O_1|S_1=1) = \pi_1 \times P(A|1) = 0.6 \times 0.4 = 0.24\\
&\alpha_1(2) = \pi_2 \times P(A|2) = 0.4 \times 0.2 = 0.08\\
&\alpha_2(1) = P(O_2|S_2=1) \times \sum_{i}a_{i1} \alpha_1(j) \\
&       = b_{1g} \times (a_{11}\alpha_1(1) + a_{21}\alpha_1(2)) =0.04\\
&\alpha_2(2) = b_{2c} \times (a_{12}\alpha_1(1) + a_{22}\alpha_1(2)) = 0.048\\
&\alpha_3(1) = b_{1c} \times (a_{11}\alpha_2(1) + a_{21}\alpha_2(2)) = 0.00944\\
&\alpha_3(2) = b_{2c} \times (a_{12}\alpha_2(1) + a_{22}\alpha_2(2)) = 0.01632\\
&\alpha_4(1) = b_{1g} \times (a_{11}\alpha_3(1) + a_{21}\alpha_3(2)) = 0.0039408\\
&\alpha_4(2) = b_{2g} \times (a_{12}\alpha_3(1) + a_{22}\alpha_3(2)) = 0.0012624\\
&\alpha_5(1) = b_{1t} \times (a_{11}\alpha_4(1) + a_{21}\alpha_4(2)) = 0.000326352\\
&\alpha_5(2) = b_{2t} \times (a_{12}\alpha_4(1) + a_{22}\alpha_4(2)) = 0.000581904\\
&\alpha_6(1) = b_{1a} \times (a_{11}\alpha_5(1) + a_{21}\alpha_5(2)) = 0.000184483\\
&\alpha_6(2) = b_{2a} \times (a_{12}\alpha_5(1) + a_{22}\alpha_5(2)) = 8.94096E-05\\		
\end{align*}
$P(O; \theta) = \alpha_6(1) + \alpha_6(2) = 0.000273893$ \\

\subsection{2 (b)}  	
\begin{align*}
\beta_{t-1}(i) &= \sum_{j=1}^2 \beta_{t}a_{ij}P(O_t|X_t=S_j)\\
&\beta_6(1) = 1\\
&\beta_6(2) = 1\\
&\beta_5(1) = \beta_6(1)a_{11}b_{1a} + \beta_6(2)a_{12}b_{2} =0.28 \\
&\beta_5(2) = \beta_6(1)a_{21}b_{1a} + \beta_6(2)a_{22}b_{2a} = 0.34\\
&\beta_4(1) = \beta_5(1)a_{11}b_{1t} + \beta_5(2)a_{12}b_{2} =  0.064\\
&\beta_4(2) = \beta_5(1)a_{21}b_{1t} + \beta_5(2)a_{22}b_{2a} =0.049
\end{align*}
\begin{align*}
				P(X_6=S_1 | O, \theta) &= \frac{\alpha_6(S_1)\beta_6(S_1)}{\alpha_6(S_1)\beta_6(S_1)+\alpha_6(S_2)\beta_6(S_2)}\\
				&= \frac{0.000184483\times  1}{0.000184483\times 1 +8.94096E-05\times 1}\\
				&= 0.673559875\\
				P(X_6=S_2 | O, \theta) &= \frac{\alpha_6(S_2)\beta_6(S_2)}{\alpha_6(S_1)\beta_6(S_1)+\alpha_6(S_2)\beta_6(S_2)}\\
				&= \frac{8.94096E-05 \times 1}{0.000184483\times 1 +8.94096E-05\times 1} \\
				&= 0.326440125
				\end{align*}

\subsection{2 (c)}  
Similarly,	

\begin{align*}
P(X_4=S_1 | O, \theta) &= \frac{\alpha_4(S_1)\beta_4(S_1)}{\alpha_4(S_1)\beta_4(S_1)+\alpha_4(S_2)\beta_4(S_2)}\\
				&= \frac{0.0039408\times  0.064}{0.0039408\times 0.064 +0.0012624\times 0.049}\\
				&= 0.705017437\\
P(X_4=S_2 | O, \theta) &= \frac{\alpha_4(S_2)\beta_4(S_2)}{\alpha_4(S_1)\beta_4(S_1)+\alpha_4(S_2)\beta_4(S_2)}\\
				&= \frac{0.0012624\times  0.049}{0.0039408\times 0.064 +0.0012624\times 0.049}\\
				&= 0.294982563
				\end{align*}
\subsection{2 (d)}
$\delta_t(j) = \max_i \delta_{t-1}(i) a_{ij} P(x_t|Z_t=s_i)$

\begin{align*}
\delta_1(1) = & \pi_1 b_{1a} =0.24 \\
\delta_1(2) = &\pi_2 b_{2a} =0.08 \\
\delta_2(1) = & b_{1c} \times max(\delta_1(1)a_{11}, \delta_1(2)a_{21}) =0.0336\\
\delta_2(2) = & b_{2c} \times max(\delta_1(1)a_{12}, \delta_1(2)a_{22}) = 0.0288\\
\delta_3(1) = &b_{1c} \times max (\delta_2(1)a_{11} + \delta_2(2)a_{21} )=0.004704\\
\delta_3(2) = &b_{1c} \times max (\delta_2(1)a_{11} + \delta_2(2)a_{21} )=0.006912\\
\delta_4(1) = &b_{1g} \times max(\delta_3(1)a_{11} + \delta_3(2)a_{21} )=0.00098784\\
\delta_4(2) = &b_{1g} \times max(\delta_3(1)a_{11} + \delta_3(2)a_{21} )=0.00041472\\
\delta_5(1) = &b_{1t} \times max(\delta_4(1)a_{11} + \delta_4(2)a_{21} )=0.0000691\\
\delta_5(2) = &b_{1t} \times max(\delta_4(1)a_{11} + \delta_4(2)a_{21} )=0.0000889\\
\delta_6(1) = &b_{1a} \times max(\delta_5(1)a_{11} + \delta_5(2)a_{21} )=0.0000194\\
\delta_6(2) = &b_{1a} \times max(\delta_5(1)a_{11} + \delta_5(2)a_{21} )=0.0000107
\end{align*}
Most likely path = arg $ max_j \delta_T(j)$ =s1,s1,s2,s1,s2,s1

\subsection{2 (e)}    	
Lets assume the $O_7 =x$ ,where $ x \in (A,C,T,G)$	
\begin{align*}
P(O_7|O) &= \sum_{i=1}^2 P(O_7,X_7=S_i|O)\\
&=  \sum_{i=1}^2 P(O_7|X_7=S_i) \times \sum_{j=1}^2 P(X_7=S_i,X_6=S_j|O)\\
&=  \sum_{i=1}^2 P(O_7|X_7=S_i) \times \sum_{j=1}^2 P(X_7=S_i|X_6=S_j) P(X_6=S_j|O)\\
& = b_{1x} \times (P(X_6=S_1|\theta)\times a_{11} + P(X_6=S_2|\theta) \times a_{21}) \\ &+ b_{2x} \times (P(X_6=S_1|\theta)\times a_{12} + P(X_6=S_2|\theta) \times a_{22})\\
\end{align*}
\begin{align*}
P(O_7=A|\theta) =&0.4\times (0.673559875 \times 0.7 +	0.326440125\times	0.4) \\
&+ 0.2\times ( 0.673559875\times	0.4 +	0.326440125\times	0.6)=0.33388479 \\
P(O_7=T|\theta) =&0.2\times (0.673559875 \times 0.7 +	0.326440125\times	0.4) \\
&+ 0.4\times ( 0.673559875\times	0.4 +	0.326440125\times	0.6)=0.306528803 \\
P(O_7=C|\theta) =&0.3\times (0.673559875 \times 0.7 +	0.326440125\times	0.4) \\
&+ 0.1\times ( 0.673559875\times	0.4 +	0.326440125\times	0.6)=0.227149191 \\
P(O_7=G|\theta) =&0.1\times (0.673559875 \times 0.7 +	0.326440125\times	0.4) \\
&+ 0.3\times ( 0.673559875\times	0.4 +	0.326440125\times	0.6)=0.199793204 \\
 \end{align*}	
We observe that the observation A  is more probable than the others,  { $P(O_7= A| O_{1:6})$ }.		 	
\end{document}